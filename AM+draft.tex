\documentclass[12pt]{article}

\linespread{1.5}
\addtolength{\oddsidemargin}{-2.5cm}
%\addtolength{\evensidemargin}{-3cm}
\addtolength{\textwidth}{2.5cm}
\begin{document}

\title{Making multiple-locus association mapping on a genome-wide scale routine}
\author{Andrew W. George and Joshua Bowden}

\maketitle


\abstract{
The methods of choice for genome wide association mapping all share a common shortcoming. 
They measure the strength of marker-trait associations on a locus by locus basis. 
More sophisticated multi-locus methods that map multiple marker-trait associations jointly are available. 
However, these methods are computationally demanding, are not always easy to implement, and can give results that are difficult to interpret. The aim of this paper is to present our method that makes the multi-locus analysis of data from genome wide association studies routine. It is formulated within a linear mixed model framework. We call our method AMplus. 
AMplus produces results faster than competing multi locus methods and often with greater statistical power. 
In terms of speed, AMplus is just a little slower than the fastest single-locus linear mixed model implementations. 
AMplus is freely available as a fully documented R package which we port as a docker binary for platform independent portability.
}


\section{Introduction}


Over the past decade, Genome wide association studies have changed considerably in both their analysis and design. Early gwas followed a case-control design. Association mapping methods were no more complicated than contingency table tests or regression. These designs though had a tendency to yield spurious findings if there was unrecognised population stratification. This prompted a shift towards family-based designs and score tests, such as the tdt test and its variants (refs), became popular. Today, instead of by design, it is through statistical modelling that we account for the effects of population stratification. This has meant that data can be collected from general populations, even if these populations are highly structured. Analysis via sophisticated association mapping methods based on linear mixed models is now almost routine.

What has not changed over the past decade is that it remains common practice to analyse gwas data on a locus by locus basis. This is despite there being several significant problems with analysing the data in this way. First, the aim of association mapping is to identify regions of the genome that house genes that are influencing a trait. The identification of these regions from these analyses is not always straightforward. Gwas results are reported, typically, via Manhattan plots that plot the -log10 of the p value for each locus against the map position of the locus. The location of peaks in this plot signify regions of interest. Inferring the exact number of regions of interest can be difficult If the peaks are not well separated. Second, when multiple statistical tests are performed, the probability of wrongly accepting a significant result (type 1 error) is inflated. This is known as the multiple testing problem (refs). Many different solutions have been offered (refs). Yet, there is still no well accepted way of correcting for multiple testing in the context of genome- wide association mapping. Third, many of the traits whose genetic secretes we are trying to discover are complex. There will be multiple loci in linkage equilibrium with genes that influence the trait. Yet, A locus by locus mapping approach only assesses the evidence for association between a single locus and trait.

It is somewhat surprising then that multiple locus methods haven't attracted more attention. It is possible to measure all locus-trait associations jointly with methods based on the regularisation techniques, ridge regression and lasso. Here, multiple testing is not an issue. These techniques though are computationally demanding. Also, the results are not easy to interpret. The strength of association is not measured by a p-value but by the size of the regression coefficient for a locus in the model. More recently, associations have started to be mapped with random forests (refs). Similar to regularisation techniques though, it is not clear how to infer genomic regions of interest from their findings (refs). An multiple locus method that does show promise is the multi-locus linear mixed model method (ref). The best multiple locus model is built with simple variable selection. Results are immediately interpretable but here, computation can be a challenge for large data.

Here, we present our new multiple locus method for genome wide association mapping, which we are calling AMplus. AMplus combines the strength of regularisation techniques (being able to fit all locus-trait associations jointly), with simple variable selection (having easy to interpret results). Our method does not require a significance threshold to be set nor regularisation parameters to be fine tuned. Through a clever dimension reduction step, we are able to achieve a computational performance similar to the fastest single locus linear mixed model implementations. AMplus is statistically more powerful than single locus association mapping and is as and often more powerful than 
most multiple locus methods. Our aim is to make multiple locus association mapping on a genome wide scale routine. To this end, we have created an R package ...


Our am+ approach is implemented in a fully documented and maintained R package. 
We are able to achieve impressive performance by moving the computationally 
components of our algorithm to Cpp. Here, we make full use of distributed 
computing, through multiple cpu and multiple glue processing when available. 
Our R package is easy to use, even for non  R users. It consists of a single function
 to read in the data and a single function for analysis. We will demonstrate that 
 our approach returns more accurate results faster and more efficiently than 
 competing multiple locus association mapping approaches. Our am+ approach 
 makes multiple locus association mapping of genome wide data routine.

\section{Results}


\subsection{Time and Memory Performance}




\subsection{Computer programs for   multiple locus association mapping}

We compared the performance of our R package, am, to six other computer programs for multiple locus association mapping. These programs are ..., ..., ... We decided on these six because they implement published methods for multiple locus association mapping.  are capable of analysing association data on a genome wide scale, and are freely available. A summary of each of these programs along with am+ is given in table X. In preparing this table, we assumed the purpose of analysing data from a genome wide association study is to identify those snp in true association with a trait.  From table X, the computer programs vary greatly in methodology, implementation, requirements, and utility.  For example, some programs have methodology that fits all marker-trait associations simultaneously where others  fit only a subset of marker-trait associations jointly. Some programs have tuning parameters where others have none. Several cannot handle additional fixed effects such as covariates to account for population structure. Also, the majority of programs considered here yield results that require additional computation before a "best" set of snp is realised. Only am+  fits all marker trait associations simultaneously, handles different input data, accommodates additional fixed effects, has no parameters to be fine tuned, returns the "best" set of snp in joint association with a trait without additional computation, and is easy to use. 


\subsection{Run times}


\subsection{Memory}


\subsection{Computational performance}
We conducted a simulation study to assess  the computational performance of am+. For comparison, we assessed six other computer programs (table X) that  also implement multiple locus association mapping. The marker data in our simulation study comes from phase 3 of the 1000 genomes project. Here, 2315 individuals were sequenced and their genotype information made available. We generated trait data by randomly selecting snp, assigning allelic  effects to the selected snp, and aggregating these effects along with random error. The number of snp  selected  was sampled from a Poisson distribution with mean 30. The effect sizes were sampled from a beta distribution. The errors were drawn from a normal distribution with zero mean and whose variance yielded a quantitative trait with a heritability of 0.5.


We were interested in how study size impacts performance. We generated replicates under five different study sizes. These were 150 individuals and 4000 snp (150 X 4K), 300 individuals and 50000 snp (300 X 50k), ... We based these five study sizes on the size of study being performed in unsequenced crop plants such as wheat, ......, respectively. For those study sizes that exceeded the number of genotyped individuals in the 1000 genomes project,  we simulated additional data by forming genotype mosaics from the real snp data. We generated 250 replicates for each scenario. 

Performance was measured by recording elapse  time and maximum memory usage. Analyses were performed on a high end desk top which was a dell intra quad core processor with 120G of RAM and a Intel gpu. Programs were run for a maximum of 100 hours.  All programs except piMass  made use of distributed computing, either implicitly via multithreaded blas and lapac linear algebra libraries or as in the case of ranfor, explicitly. For ranfor, we wrote R code to parallelise the random forest analysis across multiple cpu.  All scripts for performing these analyses are available from the Csiro-data acces portal.

In figure X we compare the elapse time of  am+ against the other computer programs. On the y axis, we have the ratio of each computer program's elapse time to the elapse time of cpu-based am+. On the X axis, we have the number of genotypes and study size. Both axes are on log scale . A positive (negative) log ratio means am+ had a shorter (longer) elapse time  than the other computer program. The magnitude (value) of the ratio is (the magnitude of)  how many times shorter or longer am+ is to the other computer program.  Since performance varies with replicate, for a given study size, a point in figure X is the log median of 250  ratios of elapse times.


From figure X, the superior performance of am+ to the other computer programs is evident. Describe ...


\subsection{Simulation study to evaluate computational performance}

\subsection{Simulation study to evaluate statistical performance}

\subsection{Application to Arabidopsis data}


\end{document}
