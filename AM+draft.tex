\documentclass[12pt]{article}

\linespread{1.5}
\addtolength{\oddsidemargin}{-2.5cm}
%\addtolength{\evensidemargin}{-3cm}
\addtolength{\textwidth}{2.5cm}
\begin{document}

\title{Making multiple-locus association mapping on a genome-wide scale routine}
\author{Andrew W. George and Joshua Bowden}

\maketitle


\abstract{Since the earliest of genome-wide association studies, a key shortcoming in how their data have been analysed has persisted. The strength of association between a marker locus and trait is measured for each locus separately, on a locus-by-locus basis. Multiple-locus methods that map multiple locus-trait associations simultaneously have been available for some time. However, they have attracted little attention. They can be demanding, computationally, and their results are not always easy to interpret. Yet, it is widely accepted that multiple-locus methods are superior, statistically, to locus-by-locus methods. Here, we present our method that makes the multiple-locus analysis of data from genome wide association studies routine. It is formulated within a linear mixed model framework. 
We call our method AMplus.  AMplus produces results faster than competing multiple-locus methods and often with greater statistical power. Also, it is just a little slower than the fastest single-locus linear mixed model implementations. 
AMplus is freely available as a fully documented R package. 
}



\section{Introduction}


Over the past decade, Genome wide association studies have changed considerably in both their analysis and design. Early gwas followed a case-control design. Association mapping methods were no more complicated than contingency table tests or regression. These designs though had a tendency to yield spurious findings if there was unrecognised population stratification. This prompted a shift towards family-based designs and score tests, such as the tdt test and its variants (refs). Today, instead of by design, it is through statistical modelling that we account for the effects of population stratification. This has meant that data can be collected from general populations, even if these populations are highly structured. Analysis via sophisticated association mapping methods based on linear mixed models is now almost routine.

What has not changed over the past decade is that it remains common practice to analyse gwas data on a locus by locus basis. This is despite there being several significant problems with analysing the data in this way. First, the aim of association mapping is to identify regions of the genome that house genes that are influencing a trait. The identification of these regions from these analyses is not always straightforward. Gwas results are reported, typically, via Manhattan plots that plot the -log10 of the p value for each locus against the map position of the locus. The location of peaks in this plot signify regions of interest. Inferring the exact number of regions of interest can be difficult If the peaks are not well separated. Second, when multiple statistical tests are performed, the probability of wrongly accepting a significant result (type 1 error) is inflated. This is known as the multiple testing problem (refs). Many different solutions have been offered (refs). Yet, there is still no well accepted way of correcting for multiple testing in the context of genome- wide association mapping. Third, many of the traits whose genetic secretes we are trying to discover are complex. There will be multiple loci in linkage equilibrium with genes that influence the trait. Yet, A locus by locus mapping approach only assesses the evidence for association between a single locus and trait.

It is somewhat surprising then that multiple locus methods haven't attracted more attention. It is possible to measure all locus-trait associations jointly with methods based on the regularisation techniques, ridge regression and lasso. Here, multiple testing is not an issue. These techniques though are computationally demanding. Also, the results are not easy to interpret. The strength of association is not measured by a p-value but by the size of the regression coefficient for a locus in the model. More recently, associations have started to be mapped with random forests (refs). Similar to regularisation techniques though, it is not clear how to infer genomic regions of interest from their findings (refs). An multiple locus method that does show promise is the multi-locus linear mixed model method (ref). The best multiple locus model is built with simple variable selection. Results are immediately interpretable but here, computation can be a challenge for large data.

Here, we present our new multiple locus method for genome wide association mapping, which we are calling AMplus. AMplus combines the strength of regularisation techniques (being able to fit all locus-trait associations jointly), with simple variable selection (having easy to interpret results). Our method does not require a significance threshold to be set nor regularisation parameters to be fine tuned. Through a clever dimension reduction step, we are able to achieve a computational performance similar to the fastest single locus linear mixed model implementations. AMplus is statistically more powerful than single locus association mapping and is as and often more powerful than 
most multiple locus methods. Our aim is to make multiple locus association mapping on a genome wide scale routine. To this end, we have created a fully documented R package, that is easy to use, even for non R users.  Our package accepts marker data of different 
formats and can handle data larger than a computer's  memory capacity. It includes detailed error checking and makes heavy use of 
distributed computing for computation when available.  The purpose of this work was to make multiple-locus association mapping on 
a genome-wide scale, for large data sets,  practical and we have built AMplus accordingly. 


\section{Results}

\subsection{Association Mapping Methods}

We compared AMplus, in terms of computational and statistical performance, against seven other association mapping methods. 
We chose methods that had been purpose built for genome-wide analysis, that could handle data from quantitative traits, and where the methods had been implemented in freely available computer software. Two of the methods are based on single locus (or locus-by-locus) models and five are based on multiple locus models. Of the many ways of performing single locus association mapping, we chose GEMMA and FaST-LMM because of their popularity and computational speed. For multiple locus association mapping, we chose bigRR, glmnet, LMM-Lasso, MLMM, and r2VIM.  Each takes a different approach to mapping multiple locus-trait associations jointly.


From a practical perspective, how the methods were implemented, in terms of usability, varied greatly. In Table 1, we list seven important features for ensuring the usability of computer software for association mapping. We also identify which software has what features. 
In forming this list of features, we assumed that the primary purpose of a genome-wide association study is 
to identify the set of marker loci in true and strongest association with a trait.  The level of documentation varied across the implementations but only 
a few came with user manuals detailing the format of the input data and how to perform analyses. 
Most implementations could handle additional fixed effects which is important when accounting for hidden population structure. 
Few though could cope with marker data larger than the memory capacity of the computer. Similarly, few returned the best set of marker loci in strongest association with a trait as their result. Almost all the software returned results that required further processing, either 
by identifying the best set of marker loci via the setting of significance thresholds or having to do further analysis to calculate the significance of the results. Almost none of the association mapping software checked for errors in the input data.  Only AMplus has all seven features. 


\subsection{Computational Performance: run times and memory usage}


To asses the computational performance of AMplus against the above discussed software, we conducted a large simulation study. 
In this study, we were interested in how study size impacts performance. We generated data under five different scenarios. 
These are a study of size 150 individuals and 5000 snp (150 X 5K), 350 individuals and 400000 snp (350 X 400K),  1500 individuals and 
50000 snp (1500 x 50K), 2000 individuals and 500000 snp (2000 x 500K), 4000 individuals and 
1500000 snp (4000 x 1.5M), and 10000 individuals and 1500000 snp (10000 x 1.5M).  These scenarios reflect some of the differences in 
study size of GWAS being performed in plants, animals, and humans.   

Under each scenario, we generated 100 replicates of data. A replicate consisted of snp and trait data. 
The snp data comes from the publically available 1000 genomes project (phase 3). The quantitative trait data were generated by randomly selecting a set of snp loci, assigning allelic effects to the selected snp, and aggregating these effects for each individual along with random error. The number of snp selected for a replicate is random, having been drawn from a Poisson distribution with mean 30. The quantitative trait was generated to have a heritability of 50\%.

Analyses were performed on a high end desktop computer with dual 8-core Xeon processors, three Kepler Tesla GPUs, and 128 Giggabytes of RAM. 
All implementations except, Gemma, made use of distributed computing for their linear algebra operations through multithreaded blas and lapac libraries.

In figure ...



We conducted a simulation study to assess  the computational performance of am+. For comparison, we assessed six other computer programs (table X) that  also implement multiple locus association mapping. The marker data in our simulation study comes from phase 3 of the 1000 genomes project. Here, 2315 individuals were sequenced and their genotype information made available. We generated trait data by randomly selecting snp, assigning allelic  effects to the selected snp, and aggregating these effects along with random error. The number of snp  selected  was sampled from a Poisson distribution with mean 30. The effect sizes were sampled from a beta distribution. The errors were drawn from a normal distribution with zero mean and whose variance yielded a quantitative trait with a heritability of 0.5.


We were interested in how study size impacts performance. We generated replicates under five different study sizes. These were 150 individuals and 4000 snp (150 X 4K), 300 individuals and 50000 snp (300 X 50k), ... We based these five study sizes on the size of study being performed in unsequenced crop plants such as wheat, ......, respectively. For those study sizes that exceeded the number of genotyped individuals in the 1000 genomes project,  we simulated additional data by forming genotype mosaics from the real snp data. We generated 250 replicates for each scenario. 

Performance was measured by recording elapse  time and maximum memory usage. Analyses were performed on a high end desk top which was a dell intra quad core processor with 120G of RAM and a Intel gpu. Programs were run for a maximum of 100 hours.  All programs except piMass  made use of distributed computing, either implicitly via multithreaded blas and lapac linear algebra libraries or as in the case of ranfor, explicitly. For ranfor, we wrote R code to parallelise the random forest analysis across multiple cpu.  All scripts for performing these analyses are available from the Csiro-data acces portal.

In figure X we compare the elapse time of  am+ against the other computer programs. On the y axis, we have the ratio of each computer program's elapse time to the elapse time of cpu-based am+. On the X axis, we have the number of genotypes and study size. Both axes are on log scale . A positive (negative) log ratio means am+ had a shorter (longer) elapse time  than the other computer program. The magnitude (value) of the ratio is (the magnitude of)  how many times shorter or longer am+ is to the other computer program.  Since performance varies with replicate, for a given study size, a point in figure X is the log median of 250  ratios of elapse times.


From figure X, the superior performance of am+ to the other computer programs is evident. Describe ...


\subsection{Simulation study to evaluate computational performance}

\subsection{Simulation study to evaluate statistical performance}

\subsection{Application to Arabidopsis data}


\end{document}
