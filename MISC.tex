\ might need this for methods section. ...

The implementations evaluated are as follows. Glmnet is a general purpose R package. It implements lasso, an estimation technique where parameter estimation and variable selection occur simultaneously, for linear models. It's value for multiple locus association mapping has been demonstrated (ref). LARS in another general purpose R package implementing methods amenable for  multiple locus association mapping (ref). It relies on least angle regression, a form of forward selection, for model selection with linear models. BigRR is purpose built R package for the genome wide analysis of association data. It is based on linear models where parameter estimation is accomplished with ridge regression, a regularisation and shrinkage technique. Lmm-lasso and MLMM are two implementations purpose built for genome wide association mapping. Lmm-lasso uses lasso to identify the set of loci in association with the trait. MLMM instead implements a forward selection strategy with backward elimination. Both are based on linear mixed models. Pimass and randfor implement two very different approaches for multiple locus association mapping. In pimass, association mapping is performed within a Bayesian framework through Bayesian variable selection.  In ranfor, assoc mapping is performed in a machine learning framework through random forests. See table X for the  key features of the implementations evaluated in this paper.

